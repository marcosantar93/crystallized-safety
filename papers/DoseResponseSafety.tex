\documentclass[11pt]{article}
\usepackage[margin=1in]{geometry}
\usepackage{graphicx}
\usepackage{booktabs}
\usepackage{amsmath}
\usepackage{hyperref}
\usepackage[nameinlink,noabbrev]{cleveref}

\title{Dose-Response Safety in Activation Steering:\\
Safety Is a Knob, Not a Diode}
\author{Marco Santarcangelo Zazzetta}
\date{\today}

\begin{document}
\maketitle

\begin{abstract}
We study activation steering for refusal behavior and show that safety failure follows a graded
dose-response curve rather than a binary diode-like transition. Using an effective steering scale,
position-aware response classification, and coherence gating, we observe four regimes:
clean refusal, educational pivot, full compliance, and collapse. The educational pivot regime is
especially concerning because harmful instructions can be embedded in refusal-style framing.
\end{abstract}

\section{Method}
We extract a steering direction from contrastive harmful/harmless prompts and apply it at a
target layer with varying steering strength. Responses are labeled by a three-way classifier:
\texttt{refusal}, \texttt{compliance}, or \texttt{collapse}. The classifier is position-aware and
explicitly handles refusal-prefixed instructional outputs.

\section{Main Result}
The measured compliance rate increases gradually with steering dose before coherence degrades,
revealing four phases:
\begin{enumerate}
  \item Clean refusal
  \item Educational pivot
  \item Full compliance
  \item Collapse
\end{enumerate}

\begin{figure}[t]
  \centering
  \includegraphics[width=0.9\linewidth]{dose_response_curve.png}
  \caption{Dose-response curve and outcome composition from the core notebook sweep.}
  \label{fig:dose-response}
\end{figure}

\begin{table}[t]
  \centering
  \begin{tabular}{@{}lll@{}}
    \toprule
    Phase & Behavior & Risk \\
    \midrule
    Clean refusal & Safe rejection & Low \\
    Educational pivot & Refusal framing + harmful instructions & High \\
    Full compliance & Direct harmful guidance & Very high \\
    Collapse & Incoherent output & Mixed \\
    \bottomrule
  \end{tabular}
  \caption{Four-phase interpretation of steering outcomes.}
  \label{tab:phases}
\end{table}

\section{Why the Diode Hypothesis Failed}
Two issues caused the earlier diode interpretation: (1) steering magnitudes that were too small to
reach the transition band and (2) classifier false positives/negatives on long refusal-framed
outputs. After fixing both, the transition is smooth and controllable.

\section{Future Work: Activation-Based Attack Detection}
Next work should add online defense by monitoring hidden-state trajectories for attack signatures
(including multi-turn prompt attacks), then triggering conditional counter-steering or policy
gating only when risk is detected.

\section{Conclusion}
Activation steering exposes a controllable safety dose-response. This enables both offensive
characterization (how safety fails) and defensive control (when and how to intervene).

\bibliographystyle{plain}
\begin{thebibliography}{9}
\bibitem{arditi2024}
Arditi et al. Refusal in Language Models Is Mediated by a Single Direction. 2024.

\bibitem{turner2023}
Turner et al. Activation Addition: Steering Language Models Without Optimization. 2023.

\bibitem{korznikov2025}
Korznikov et al. The Rogue Scalpel: Activation Steering Compromises LLM Safety. 2025.

\bibitem{tan2025}
Tan et al. Programming Refusal with Conditional Activation Steering. 2025.
\end{thebibliography}

\end{document}
